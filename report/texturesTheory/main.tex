\section{Mapping Transformations}

A forward mapping is a transformation applied to a matrix resulting in a
spatial correspondences between all points in the matrix to a warped
matrix with the same dimensions. Spatial here means that each point in the
original matrix is affected and has a "warped" counterpart in the other
matrix. This transformation can be expressed in two "directions" Input going
to output, and output going to input. 
To make notation easier we will use the notation $(x,y)$ to denote a
coordinates in the output image and $(u,v)$ to denote coordinates in the input
image. We will further define the mapping function X,Y,U and V which will map a
value to it's lowercase value. 

With these definitions we can describe the two kinds of mappings (forward and
inverse) as:

$$[x, y] = [X (u,v),Y(u, v)]$$

and

$$[u,v] = [U(x, y),V (x, y)]$$

Using this, we can transform matrices though the properties of
homographies described previously
