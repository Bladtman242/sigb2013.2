%\item properties (transforms between planes, associative (applied as functions)
\subsection{Properties}
Representing operations on points as matrix multiplications makes it possible
to use the matrices as functions. It is worth mentioning that while matrix
multiplication is associative, it not commutative. Naturally, the same goes for
homographies. I.e. for some homograhies $H$, $H1_{1}$ and some point $p$ $$H*p
\ne p*H$$ and $$H*(H_{1}*p) = (H*H_{1})*p$$ The feature of associativity can,
computationally speaking, be a time saver. Often it is useful to apply several
homographies e.g. one for rotation and one for translation. Rather than
applying both homographies to all relevant points p $$p' = H_{r} * (H_{t} * p)$$ one
can, equivalently, pre compute a new homography $H=H_{r}*H_{t}$ and use it as
$$p' = H * p$$ for all points p.

It also means that for projections between 3 planes, $G, M, T$, and
homographies $H_{G}^{M}$ from $G$ to $M$, $H_{M}^{T}$ from $M$ to $T$, the
following holds true: $$H_{G}^{T} = H_{G}^{M} * H_{M}^{T}$$.

Another important property of homographies is invertibility. Any tranformation
from one plane to another can be inversed. Homography invertibility translates
directly to matrix invertibility, so that $H_{T}^{G} = (H_{G}^{T})^{-1}$
Thus we can expand the previous example with
$$H_{T}^{G} = (H_{G}^{M} * H_{M}^{T})^{-1}$$
