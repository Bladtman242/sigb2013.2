%\item representation (homogeneity)
\subsection{Representation}
The operations described in the previous section can be neatly packaged into
matrix form:
$$\begin{bmatrix}
	r,sc & r,sh & t_{x} \\
	r,sh & r,sc & t_{y} \\
	h_{31} & h_{32} & hom
\end{bmatrix}$$

Where $r,sc$ are combined rotation and scaling parameters, $r,sh$ are combined
rotation and shearing parameters, $t_{x}$ and $t_{y}$ are translation
parameters for the x and y axis respectively, and $hom$ is there to accommodate
homogeneity. 

Homogeneous coordinates in an $n$-dimensional space is represented as a vector
in $n+1$-dimensional space. So a two-dimensional, Cartesian point in Euclidean
geometry
$$\begin{pmatrix}
	x \\
	y
\end{pmatrix}$$
is represented as the three-dimensional, homogeneous point
$$\begin{pmatrix}
	x' \\
	y' \\
	z
\end{pmatrix}$$
in projective geometry, such that $$x'=x*z,$$$$y'=y*z$$To convert homogeneous
coordinates to Cartesian ones,simply take
$$\begin{pmatrix}
	\frac{x}{z} \\[0.3em]
	\frac{y}{z} \\[0.3em]
	z
\end{pmatrix}$$
The homogeneous representation allows multiplication with a (non-zero) scalar,
without changing the position of the coordinate. Perhaps more notably in this
context, it allows $n$-dimensional points to be multiplied with $(n+1)\times
(n+1)$ matrices.

