\subsection{Taking it live}

As soon as the method for augmenting a box on a single frame is in place,
expanding to live view is a pretty easy task. 
In the old augmentation method we did a fair share of redundant
calculations. In order for the live augmentation to run smoothly, these had
to be moved out so they’ll only be done once.

Reading in the pattern image, converting it to greyscale and loading the
calibration-matrixes were moved out to global vars instead.

\begin{verbatim}
I1 = cv2.imread("pattern.png")
I1Gray = cv2.cvtColor(I1,cv2.COLOR_RGB2GRAY)
K, dist_coefs = calibrate.loadMatrixes()
\end{verbatim}

Calculating the corderns of the original pattern image, getting the box cords
and initialising the first camera instance also only needed to be done once.

\begin{verbatim}
I1Corners = getCornerCoords(I1Gray)
box = cube_points((0,0,0),0.3)
cam1 = Camera(hstack((K,dot(K,array([[0],[0],[-1]])) )) )
\end{verbatim}

Notice that these are only minor speed improvements, as each frame still has
some heavy calculations to be made:

\subsubsection{We still have to}
Find the chessboard corners on the supplied frame

\begin{verbatim}
I2Corners = getCornerCoords(I2Gray)
\end{verbatim}

Calculate a homography between the two chessboard 2d surfaces.

\begin{verbatim}
H = estimateHomography(I1Corners, I2Corners)
\end{verbatim}

Project the box cords onto the new surface.

\begin{verbatim}
box_cam2 = np.array(cam2.project(toHomogenious(box)))
\end{verbatim}

And then draw the box in the frame.
The end result could of course be improve further (as always), but for the scope
of this assignment we think the result is very acceptable, and undeniably cool
:-)
See. Aug.avi for the result video.
