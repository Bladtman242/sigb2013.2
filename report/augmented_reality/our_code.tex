\subsection{Our Code: (the important lines)}

\begin{verbatim}
//This line initializes the first camera, with K * [I | O] .

 cam1 = Camera(hstack((K,dot(K,array([[0],[0],[-1]])) )) )

//We estimate the homography “H” between the two chessboards by detecting 4 corners in each image.

H = estimateHomography(I1Corners, I2Corners)

//The homography is used on the existing (frontal) camera

  cam2 = Camera(dot(H,cam1.P))

//We isolate the [Rotation | translation] matrix by multiplying by inverse of “K” calibration matrix.

 A = dot(linalg.inv(K),cam2.P[:,:3]) 

//The intrinsic parameters are initialized. The X,Y axes are already valid after being multiplied by the homography matrix, and the third column vector is estimated with the cross product.

    A = array([A[:,0],A[:,1],np.cross(A[:,0],A[:,1],axis=0)]).T

// The calibration matrix is added again by multiplication, and the result is stored in the new camera.

    cam2.P[:,:3] = np.dot(K,A[0])

 //The box coordinates are projected onto the plane with the second camera

box_cam2 = np.array(cam2.project(toHomogenious(box)))
\end{verbatim}

Note: Concerning getting a better homography we found that using the
original pattern image “pattern.png” instead of our own frontal webcam
photo produced a better result. It proved hard to capture a frontal image
without any rotation or translation. At least with using the pattern image we
avoid rotations, but a translation is still present as the pattern is not
completely centered. 
